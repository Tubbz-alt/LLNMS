\documentclass[12pt]{report}


\begin{document}

\section*{Design}

\subsection*{Overview}
The chief objective with LLNMS is \emph{modularity} and
\emph{independence}.  Network Management Systems (NMS) tend
to consist of a single application which does all of the work.  
LLNMS seeks to be a collection of tools which chained together 
designate a complete NMS solution. 


\subsubsection*{Networks}

LLNMS allows you to scan a network to view all available or detected assets.  This 
can be used to monitor the health of a network infrastructure or to detect intruders into
a system.  To implement this capability, LLNMS scans a network using a specified set of 
rules.  

\section*{Hierarchy}

LLNMS operates as an independent set of utilities which depend on 
a deterministic file and directory structure. 

\subsubsection*{LLNMS\_HOME}
This is the base directory for which all LLNMS activity takes place.  This
contains all configuration files, utilities, and output.  On linux systems, 
the designated home is at \texttt{/var/tmp/llnms}. 

\subsubsection*{LLNMS\_HOME/bin}
This is the binary and script directory for all executable programs in LLNMS.  This
should be added to your path in order to ensure that LLNMS is available to the user.

\subsubsection*{LLNMS\_HOME/networks}
This is the location where all network definitions are stored.

\section*{Installation}

\subsection*{Linux}

\subsubsection*{LLNMS-NetView}
In order to install LLNMS-NetView onto your system, you will need node.js installed. In Ubuntu apt, 
this can be done via \texttt{apt-get install nodejs}.

\end{document}
